%----------------------------------------------------
% Required Packages
%----------------------------------------------------
% Basically standard AMS packages
\usepackage{amsmath, amsthm, amsfonts, amssymb}

% Better, more flexible command definitions
\usepackage{xparse}

% Math constructs defined
\usepackage{commath}

% Set the math font if using XeLatex
\ifxetex 
    \setmathfont{Cambria Math}
\else\fi

%----------------------------------------------------
%----------------------------------------------------
% Command Definitions
%----------------------------------------------------
%----------------------------------------------------

%
% Convenience command to outline a drawing canvas.
% Not actually math, but not enough non-math commands to justify a separate package
%
\newcommand{\drawframe}[2]{\put(0,0){\line(1,0){#1}}\put(0,#2){\line(1,0){#1}}\put(0,0){\line(0,1){#2}}\put(#1,0){\line(0,1){#2}}}

%
% Convenience commands for wrapping arguments in properly sized delimiters
%
\newcommand{\wrap}[1]{\left( #1 \right)}
\newcommand{\wrapP}[1]{\left( #1 \right)}
\newcommand{\wrapB}[1]{\left[ #1 \right]}
\newcommand{\wrapBr}[1]{\left\{ #1 \right\}}
\newcommand{\wrapAbs}[1]{\left| #1 \right|}
\newcommand{\wfrac}[2]{\wrap{\frac{#1}{#2}}}
\newcommand{\ldel}[1]{\left( #1 \right.}
\newcommand{\rdel}[1]{\left. #1 \right)}

%
% Short-cut commands for math fonts
%
\newcommand{\mc}[1]{\mathcal{#1}}
\newcommand{\mb}[1]{\mathbf{#1}}
\newcommand{\mbb}[1]{\mathbb{#1}}

%
% Simple math symbols
%
\DeclareMathOperator{\imp}{\Rightarrow}
\DeclareMathOperator{\contra}{\rightarrow\leftarrow}
\DeclareMathOperator{\myEquiv}{\Leftrightarrow}

\DeclareMathOperator{\reals}{\mathbb{R}}
\DeclareMathOperator{\nat}{\mathbb{N}}
\DeclareMathOperator{\ints}{\mathbb{Z}}

\DeclareMathOperator{\J}{\mathbf{J}}
\DeclareMathOperator*{\esssup}{ess\,sup}

\DeclareMathOperator{\Let}{\text{Let }}
\DeclareMathOperator{\Define}{\text{Define }}
\DeclareMathOperator{\Then}{\text{Then }}
\DeclareMathOperator{\Suppose}{\text{Suppose }}

\DeclareMathOperator{\vleq}{\rotatebox{90}{\geq}}
\DeclareMathOperator{\vgeq}{\rotatebox{270}{\leq}}
\DeclareMathOperator{\vless}{\raisebox{2ex}{\rotatebox{-90}{<}}}
\DeclareMathOperator{\vgreater}{\raisebox{2ex}{\rotatebox{-90}{>}}}

%
% Functions with mandatory arguments, technically parentheses
%
\newcommand{\erf}[1]{\text{erf}\wrap{#1}}
\newcommand{\erfc}[1]{\text{erfc}\wrap{#1}}
\newcommand{\erfcx}[1]{\text{erfcx}\wrap{#1}}

%
% Functions/Operations with optional parentheses
%
\NewDocumentCommand\cov   {g}{\text{cov} \IfValueT{#1}{\del{#1}}}
\NewDocumentCommand\var   {g}{\text{var} \IfValueT{#1}{\del{#1}}}
\NewDocumentCommand\rank  {g}{\text{rank} \IfValueT{#1}{\del{#1}}}
\NewDocumentCommand\pow   {g}{\mathcal{P} \IfValueT{#1}{\del{#1}}}
\NewDocumentCommand\median{g}{\text{median} \IfValueT{#1}{\del{#1}}}
\NewDocumentCommand\mean  {g}{\text{mean} \IfValueT{#1}{\del{#1}}}
\NewDocumentCommand\diag  {g}{\text{diag} \IfValueT{#1}{\del{#1}}}

%
% Operations with an optional conditional structure
%
\NewDocumentCommand\E    {gg}{\text{E}\IfValueT{#1}{\IfValueTF{#2}{\left\( \left. #1 \right| #2 \right\)}{\del{#1}}}}
\NewDocumentCommand\myPr {gg}{\text{Pr}\IfValueT{#1}{\IfValueTF{#2}{\left\( \left. #1 \right| #2 \right\)}{\del{#1}}}}

\RenewDocumentCommand\set  {mg}{\IfValueTF{#2}{\left\{ \left. #1 \right| #2 \right\}}{\cbr{#1}}} % Prefer my set command to 'commath' version


%
% Mathematical Structures
%
\newcommand{\ubar}[1]{\underline{#1}}

\newcommand{\floor}[1]{\left\lfloor #1 \right\rfloor}

\newcommand{\myLim}[2]{\lim_{#1 \rightarrow #2}}
\newcommand{\myMin}[1]{\genfrac{}{}{0pt}{2}{\text{min}}{#1}}
\newcommand{\myMax}[1]{\genfrac{}{}{0pt}{2}{\text{max}}{#1}}
\newcommand{\myArgMin}[1]{\genfrac{}{}{0pt}{2}{\text{argmin}}{#1}}
\newcommand{\myArgMax}[1]{\genfrac{}{}{0pt}{2}{\text{argmax}}{#1}}

\newcommand{\convergeP}{\stackrel{p}{\rightarrow}}
\newcommand{\limP}{\stackrel{p}{\rightarrow}}

\newcommand{\binomBr}[2]{\wrapBr{\genfrac{}{}{0pt}{2}{#1}{#2}}}
\newcommand{\funcDef}[3]{#1:#2\rightarrow #3}
\newcommand{\interior}[1]{\genfrac{}{}{0pt}{2}{\circ}{#1}}

\newcommand{\diff}[2]{\frac{\partial #1}{\partial #2}}
\newcommand{\Diff}[2]{\frac{\mathrm{d} #1}{\mathrm{d} #2}}
\newcommand{\secDiff}[2]{\frac{\partial^2 #1}{\partial #2^2}}
\newcommand{\secDiffM}[3]{\frac{\partial^2 #1}{\partial #3 \partial #2}}
\newcommand{\kDiff}[3]{\frac{\partial^{#3} #1}{\partial #2^{#3}}}

\NewDocumentCommand\evalBar {mgg}{\left. #1 \right|\IfValueT{#2}{_{#2}}\IfValueT{#3}{^{#3}}}

%
% Theorem-like environments
%
\@ifundefined{definition} {\newtheorem {definition} {Definition}  \newtheorem*{definition*} {Definition}  \def\definitionautorefname  {Definition}} {}
\@ifundefined{hypothesis} {\newtheorem {hypothesis} {Hypothesis}  \newtheorem*{hypothesis*} {Hypothesis}  \def\hypothesisautorefname  {Hypothesis}} {}
\@ifundefined{proposition}{\newtheorem {proposition}{Proposition} \newtheorem*{proposition*}{Proposition} \def\propositionautorefname {Proposition}} {}
\@ifundefined{lemma}      {\newtheorem {lemma}      {Lemma}       \newtheorem*{lemma*}      {Lemma}       \def\lemmaautorefname       {Lemma}} {}
\@ifundefined{corollary}  {\newtheorem {corollary}  {Corollary}   \newtheorem*{corollary*}  {Corollary}   \def\corollaryautorefname   {Corollary}} {}
\@ifundefined{note}       {\newtheorem {note}       {Note}        \newtheorem*{note*}       {Note}        \def\noteautorefname        {Note}} {}
\@ifundefined{theorem}    {\newtheorem {theorem}    {Theorem}     \newtheorem*{theorem*}    {Theorem}     \def\theoremautorefname     {Theorem}} {}
\@ifundefined{assumption} {\newtheorem {assumption} {Assumption}  \newtheorem*{assumption*} {Assumption}  \def\assumptionautorefname  {Assumption}} {}
\@ifundefined{observation}{\newtheorem {observation}{Observation} \newtheorem*{observation*}{Observation} \def\observationautorefname {Observation}} {}
\@ifundefined{result}     {\newtheorem {result}     {Result}      \newtheorem*{result*}     {Result}      \def\resultautorefname      {Result}} {}
\@ifundefined{notation}   {\newtheorem {notation}   {Notation}    \newtheorem*{notation*}   {Notation}    \def\notationautorefname  {Notation}}   {}
\@ifundefined{notation*}  {\newtheorem {notation*}  {Notation}    \newtheorem*{notation*}   {Notation}    \def\notationautorefname  {Notation}}   {}

%
% Sub-environments for proofs
%
\newenvironment{claim}
{\noindent\ignorespaces \textsc{Claim:} }
{\par\noindent\ignorespacesafterend}

\newenvironment{claimProof}
{\noindent\ignorespaces\raisebox{.5ex}{\rule{0.75ex}{0.75ex}}\hspace{1ex}\textit{\textsc{Proof of Claim:}} }
{\raisebox{.5ex}{\rule{0.75ex}{0.75ex}}\par\noindent\ignorespacesafterend}






